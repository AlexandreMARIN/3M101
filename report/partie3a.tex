%partie 3a
\paragraph{Trafic automobile :}avant de donner les r\'esultats on introduit quelques notations.

\begin{definition}
On d\'efinit quatre arborescences visibles en annexe \ref{arbo}, not\'ees $T_1$, $T_2$, $T_3$ et $T_4$.
On note \csg{p}{$\sum T_k$} le circuit de longueur $p$ dont chaque sommet est racine d'une arborescence dont les sous-arborescences sont les $T_k$.
\end{definition}

On observe alors :
\begin{figure}[h]
\begin{center}
\begin{tabular}{|c||c|}\hline
$n$ & graphes\\\hline
2 & 2\cycle{1} \cycle{2}\\\hline
3 & 2\cycle{1} 2\cycle{3}\\\hline
4 & 2\cycle{1} 2\cycle{4} \csg{2}{$2T_{1}$}\\\hline
5 & 2\cycle{1} 2\cycle{5} 2\csg{5}{$T_{1}$}\\\hline
6 & 2\cycle{1} 2\cycle{3} 2\cycle{6} 2\csg{6}{$T_{1}$} \csg{2}{$T_{2}$}\\\hline
7 & 2\cycle{1} 2\cycle{7} 2\csg{7}{$T_{1}$} 2\csg{7}{$T_{1}+T_{2}$}\\\hline
8 & 2\cycle{1} 2\cycle{4} 4\cycle{8} 4\csg{8}{$T_{1}$} 2\csg{8}{$T_{1}+T_{2}$} \csg{2}{$2T_{1}+4T_{3}$}\\\hline
9 & 2\cycle{1} 2\cycle{3} 6\cycle{9} 6\csg{9}{$T_{1}$} 2\csg{9}{$T_{1}+T_{2}$} 2\csg{9}{$2T_{1}+T_{2}+T_{3}$}\\\hline
10 & 2\cycle{1} 2\cycle{5} 2\csg{5}{$3T_{1}$} 8\cycle{10} 8\csg{10}{$T_{1}$} 4\csg{10}{$T_{1}+T_{2}$}\\
& 2\csg{10}{$2T_{1}+T_{2}+T_{3}$} \csg{2}{$5T_{1}+5T_{4}$}\\\hline
\end{tabular}
\end{center}
\caption{graphe de $f$ dans le cas du trafic automobile}
\end{figure}