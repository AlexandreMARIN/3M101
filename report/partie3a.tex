%partie 3a 
\paragraph{Exemple lin\'eaire :} Un programme python dont le code est joint \`a ce rapport nous donne le tableau suivant: \newline

\begin{figure}[h]
\begin{center}
\begin{tabular}{||c  l||}
\hline
$n$ & composantes \\ [0.5ex] 
\hline\hline
2 & $(O_1*T_4)$\\ 
\hline
3 & $(O_3*T_2)+(O_1*T_2)$ \\
\hline
4 & $(O_1*T_{16})$\\
\hline
5 & $(O_{15}*T_2)+(O_1*T_2)$ \\
\hline
6 & $2(O_6*T_4)+(O_3*T_4)+(O_1*T_4)$ \\ 
\hline
7 & $9(O_7*T_2)+(O_1*T_2)$ \\
\hline
8 & $(O_1*T_{256})$ \\
\hline
9 & $4(O_{63}*T_2)+(O_3*T_2)+(O_1*T_2)$ \\
\hline
10 & $8(O_{30}*T_4)+(O_{15}*T_4)+(O_1*T_4)$ \\
\hline
11 & $3(O_{341}*T_2)+(O_1*T_2)$ \\
\hline
12 & $20(O_{12}*T_{16})+2(O_6*T_{16})+(O_3*T_{16})+(O_1*T_{16})$ \\
\hline
13 & $5(O_{819}*T_2)+(O_1*T_2)$ \\
\hline
14 & $288(O_{14}*T_4)+9(O_7*T_4)+(O_1*T_4)$ \\
\hline
15 & $1091(O_{15}*T_2)+3(O_5*T_2)+(O_3*T_2)+(O_1*T_2)$ \\
\hline
16 & $(O_1*T_{65536})$ \\
\hline
17 & $256(O_{255}*T_2)+3(O_{85}*T_2)+(O_1*T_2)$ \\
\hline
18 & $518(O_{126}*T_4)+4(O_{63}*T_4)+2(O_6*T_4)+(O_3*T_4)+(O_1*T_4)$ \\
\hline
\end{tabular}
\end{center}
\caption{graphe de $A$ en fonction de $n$}\label{tablingraph}
\end{figure}

Ce tableau nous permet de faire de nombreuses conjectures et nous inspire \`a montrer quelques th\'eor\`emes.

L'\'etude des cycles est la derni\`ere composante qui nous manque pour conna\^itre la structure de toutes les composantes connexes du graphe de $A$.

Commen\c cons par une observation directement inspir\'ee du tableau de la figure \ref{tablincyc} . Pour chaque taille du cycle $t$ qu'on observe chronologiquement dans le tableau, on regarde les $n$ pour lesquels un cycle de tel taille appartient au graphe ($n \leqslant 40$).

\begin{figure}[!]
\begin{center}
 \begin{tabular}{||c  l  c||}
 \hline
 $t$ & $n$ & $d$ \\ [0.5ex] 
 \hline\hline
 1 & $3,4,5,6,\dots ,40$ & 1\\ 
 \hline
 3 & $3,6,9,12,15,\dots,39$ & 3 \\
 \hline
 15 & $5,10,15,20,\dots,40$ & 5\\
 \hline
 6 & $6,12,18,24,30,36$ & 6\\
 \hline
 7 & $7,14,21,28,35$ & 7\\
 \hline
 63 & $9,18,21,27,36$ & 9\\
 \hline
 30 & $10,20,30,40$ & 10\\
 \hline
 341 & $11,22,33$ & 11\\
 \hline
 12 & $12,24,36$ & 12\\
 \hline
 819 & $13,26,35,39$ & 13\\
 \hline
 14 & $14,28$ & 14\\
 \hline
 5 & $15,30$ & 15\\
 \hline
 85 & $17,34$ & 17\\
 \hline
 255 & $17,34$ & 17\\
 \hline
 126 & $18,36$ & 18\\
 \hline
\end{tabular}
\end{center}
\caption{d\'ependance entre $t$ et $n$}\label{tablincyc}
\end{figure}


On remarque que pour chaque $t$ les $n$ correspondants forment une suite arithm\'etique (sauf quelques exceptions pour $t=63$ et $t=819$). Cependant, m\^eme pour les exceptions, la suite arithmétique reste bien une sous-suite. On a not\'e par $d$ la raison de cette suite.

Regardons la taille du plus grand cycle $c$ dans le tableau de la figure \ref{tablincyc2} :

\begin{figure}[!]
\begin{center}
 \begin{tabular}{||c  l  c||} 
 \hline
 $n$ & $c$ & $c/n$ \\ [0.5ex] 
 \hline
 2 & 1 & ---\\ 
 \hline
 3 & 3 & 1\\ 
 \hline
 4 & 1 & ---\\ 
 \hline
 5 & 15 & 3\\ 
 \hline
 6 & 6 & 1\\ 
 \hline
 7 & 7 & 1\\ 
 \hline
 8 & 1 & ---\\ 
 \hline
 9 & 63 & 7\\ 
 \hline
 10 & 30 & 3\\ 
 \hline
 11 & 341 & 31\\ 
 \hline
 12 & 12 & 1\\ 
 \hline
 13 & 819 & 63\\ 
\hline
\end{tabular}
\end{center}
\caption{d\'ependance entre $n$ et $c$}\label{tablincyc2}
\end{figure}

%%%%%%%%%%%%%%%%%%%%%%%%%%

\paragraph{Trafic automobile :}avant de donner les r\'esultats on introduit quelques notations.

\begin{definition}
On d\'efinit quatre arborescences visibles en annexe \ref{arbo}, not\'ees $\mathcal{T}_1$, $\mathcal{T}_2$, $\mathcal{T}_3$ et $\mathcal{T}_4$.
On note \csg{p}{$\sum \mathcal{T}_k$} le circuit de longueur $p$ dont chaque sommet est racine d'une arborescence dont les sous-arborescences sont les $\mathcal{T}_k$.
\end{definition}

On observe alors :
\begin{figure}[h]
\begin{center}
\begin{tabular}{|c||c|}\hline
$n$ & graphes\\\hline
2 & 2\cycle{1} \cycle{2}\\\hline
3 & 2\cycle{1} 2\cycle{3}\\\hline
4 & 2\cycle{1} 2\cycle{4} \csg{2}{$2\mathcal{T}_{1}$}\\\hline
5 & 2\cycle{1} 2\cycle{5} 2\csg{5}{$\mathcal{T}_{1}$}\\\hline
6 & 2\cycle{1} 2\cycle{3} 2\cycle{6} 2\csg{6}{$\mathcal{T}_{1}$} \csg{2}{$\mathcal{T}_{2}$}\\\hline
7 & 2\cycle{1} 2\cycle{7} 2\csg{7}{$\mathcal{T}_{1}$} 2\csg{7}{$\mathcal{T}_{1}+\mathcal{T}_{2}$}\\\hline
8 & 2\cycle{1} 2\cycle{4} 4\cycle{8} 4\csg{8}{$\mathcal{T}_{1}$} 2\csg{8}{$\mathcal{T}_{1}+\mathcal{T}_{2}$} \csg{2}{$2\mathcal{T}_{1}+4\mathcal{T}_{3}$}\\\hline
9 & 2\cycle{1} 2\cycle{3} 6\cycle{9} 6\csg{9}{$\mathcal{T}_{1}$} 2\csg{9}{$\mathcal{T}_{1}+\mathcal{T}_{2}$} 2\csg{9}{$2\mathcal{T}_{1}+\mathcal{T}_{2}+\mathcal{T}_{3}$}\\\hline
10 & 2\cycle{1} 2\cycle{5} 2\csg{5}{$3\mathcal{T}_{1}$} 8\cycle{10} 8\csg{10}{$\mathcal{T}_{1}$} 4\csg{10}{$\mathcal{T}_{1}+\mathcal{T}_{2}$}\\
& 2\csg{10}{$2\mathcal{T}_{1}+\mathcal{T}_{2}+\mathcal{T}_{3}$} \csg{2}{$5\mathcal{T}_{1}+5\mathcal{T}_{4}$}\\\hline
\end{tabular}
\end{center}
\caption{graphe de $f$ dans le cas du trafic automobile}
\end{figure}