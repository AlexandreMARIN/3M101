%conclusion
\conclusion

Dans l’\'etude de ces deux dynamiques, on a r\'eussi \`a trouver certaines r\'egularit\'es dans leurs structures quand on fait varier $n$, reli\'e au cardinal du système. Dans les deux cas, les parties arborescentes pr\'esentent une forte sym\'etrie qui simplifie consid\'erablement leur \'etude. Les cycles, d’autre part, sont plus complexes dans le cas de l'exemple lin\'eaire et sauf certaines observations on n’a pas r\'eussi \`a r\'esoudre totalement le problème de leur taille. En ce qui concerne le trafic routier, les arborescences sont plut\^ot difficiles \`a d\'ecrire et la formulation des r\'esultats s'en trouve compliqu\'ee. Il nous semble que malgr\'e le caractère parfois assez r\'egulier des cycles et arborescences il y a quelque chose d’al\'eatoire dans leur comportement qui les rend incompr\'ehensibles. Comme le c\'elèbre Paul Erd\"os a dit \`a propos de la conjecture de Syracuse qui est un autre exemple de dynamique: \og{}~les math\'ematiques ne sont pas encore prêtes pour de tels problèmes~\fg{}.

On dispose en r\'esum\'e :
\begin{itemize}
\item de programmes pour construire et analyser les graphes qui nous int\'eressent ici ;
\item de preuves concernant des propri\'et\'es sur ces m\^emes graphes ;
\item d'outils pour afficher de tels graphes (on a utilis\'e les ressources \cite{GV} et \cite{NX}).
\end{itemize}