%introduction
\intro

Il s'agit ici d'\'etudier les it\'erations d'une application $f$ d'un ensemble fini $E$ dans lui-m\^eme. Pour cela on cherche \`a construire des graphes orient\'es qui permettront soit de corroborer certaines hypoth\`eses soit d'aiguiser davantage notre intuition. Ces graphes repr\'esenteront l'application $f$ en question et sont d\'efinis de la fa\c con suivante : les sommets sont les \'el\'ements de $E$ et les arcs relient chaque sommet \`a leur image par $f$.

Ces graphes sont tr\`es particuliers : on sait par exemple que chaque composante connexe comporte exactement un circuit et que chaque sommet d'un circuit est racine d'une arborescence. L'objectif est donc d'\emph{observer} ces graphes apr\`es construction avec l'outil informatique puis d'en d\'eduire des \emph{conjectures} sur certaines propri\'et\'es de ces graphes comme les longueurs des circuits.

On souhaite aussi d\'eterminer la nature de la d\'ependance de ces propri\'et\'es vis-\`a-vis du cardinal de $E$. Les algorithmes employ\'es seront plus ou moins \og na\"ifs \fg{} dans un premier temps, ce qui engendrera un temps de calcul tr\`es long lorsque le cardinal de $E$ sera trop grand. Ainsi on d\'emontrera au passage quelques r\'esultats afin d'am\'eliorer les performances des programmes cr\'e\'es.