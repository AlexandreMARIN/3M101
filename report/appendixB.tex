%appendice B
\chapter{Programmation}

\begin{rem}[astuce -- trafic automobile]
En C/C++, on peut repr\'esenter une s\'equence de $n$ bits par un entier non sign\'e de type \verb+uint32_t+ dont on ne consid\`ere que les $n$ premiers bits \`a droite dans l'\'ecriture en base $2$. On peut alors coder $f$ facilement en utilisant des op\'erations de \og masques \fg{} (symboles \verb+&+, \verb+|+, \verb+!+, \verb+^+) et de d\'ecalage de bits (symboles \verb+<<+ et \verb+>>+) qui correspondent respectivement \`a des op\'erations logiques bit-\`a-bit et \`a des divisions/multiplications par des puissances de $2$. Nous renvoyons au code de la fonction $f$ qui tient en une seule ligne.
\end{rem}

L'exemple lin\'eaire et le trafic routier sont \'etudi\'es par des programmes \'ecrits respectivement en Python et en C/C++.