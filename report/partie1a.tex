%partie I a
\begin{definition}[graphe d'une application]
Soit $E$ un ensemble fini et $f:E\rightarrow E$. Le graphe $G_f$ de l'application $f$ est le couple $(E, A)$, o\`u $A$ est l'ensemble des arcs reliant $x$ \`a $f(x)$ pour tout $x$ dans $E$.
\end{definition}

Dans la suite on consid\`ere maintenant une telle fonction $f$ et son graphe.

\begin{definition}[arborescence]
Un sous-graphe de $G_f$ est une arborescence si en inversant le sens de chacun de ses arcs, on obtient une arborescence classique au sens de la th\'eorie des graphes.
\end{definition}

\begin{definition}[homog\'en\'eit\'e]
Une composante connexe de $G_f$ est dite \emph{homog\`ene} si elle contient un unique circuit et si toutes les arborescences (ayant un nombre maximal de sommets dont un seulement appartient au circuit) qui y sont attach\'ees sont \'egales (structurellement).
\end{definition}

\begin{rem}
Dans la suite on confondra les termes suivants :
\begin{itemize}
\item ar\^ete et arc;
\item cycle et circuit;
\item arbre et arborescence.
\end{itemize}
Le vocabulaire de la th\'eorie des graphes non orient\'es suffira.
\end{rem}