%partie 3b
\begin{conj}[trafic automobile]
Les longueurs des circuits sont les diviseurs de $n$.
\end{conj}

\begin{conj}[trafic automobile]
Les circuits simples et les composantes connexes de la forme \csg{n}{$T_1$} sont pr\'esentes pour tout $n\geqslant 5$.
\end{conj}

\begin{conj}[trafic automobile]
Il y a un unique circuit de longueur 2 si et seulement si $n$ est pair. La hauteur des arborescences attach\'ees \`a ce circuit vaut $(n/2) - 1$ et le nombre de n\oe uds de la composante connexe vaut $C^{n/2}_{n}$.
\end{conj}

\begin{conj}[trafic automobile]
La composante \csg{n}{$T_{1}+T_{2}$} est pr\'esente si $n\geqslant 7$.
\end{conj}

\begin{conj}[trafic automobile]
On peut construire par r\'ecurrence une suite d'arborescences $(\mathbb{T}_{n})_n$ dont les quatre premiers termes sont $T_1$, $T_2$, $T_3$ et $T_4$ de fa\c con \`a ce que toute arborescence attach\'ee \`a un circuit ait pour sous-arborescences uniquement des \'el\'ements de $(\mathbb{T}_{n})_n$.
\end{conj}